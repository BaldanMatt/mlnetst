
% Section will be
% 1. Spatial Transcriptomics: what is it?
% 2. Imaging-based Spatial Transcriptomics and its challenges
% 3. Integration between im-scRNAseq and scRNAseq
% 4. Cell regulatory mechanisms and GRNs
% 5. Estimating the activity of biologically relevant features
% 6. Results of spatially emergent patterns.
% 7. Cell-Cell communication background with analogy with 
%    antennas that send and receive at different frequencies and
%    cells that subscribe to different frequencies as if they were
%    subscribing to differnet social networks.
% 8. Modeling the cell-cell communication system with Multilayer networks
% 9. Structural analysis and tensor decomposition of the multilayer network
% 10.Appendix:
%    - Vignette of N2SIMBA
%    - Vignette of metagenomics analysis
% 11.Pubblications, teaching and courses
\section{Spatial transcriptomics: what is it? {[01:30]}}

\begin{frame}
    \frametitle{Introduction to Spatial Transcriptomics}
    \begin{block}{}
        \begin{itemize}
            \item History of how single cell measurements went from bulk to capture the state of individual cells in space.
            \item Reference to the Nature methods awards.
            \item Analogy and definition of biological terms that will be used in the talk:
                \begin{itemize}
                    \item \textbf{Cell}: a biological unit that can be measured.
                    \item \textbf{Feature}: a biological measurement of a cell, such as a gene expression level.
                    \item \textbf{Cell type}: a group of cells that share similar features.
                    \item \textbf{Cell state}: the condition of a cell at a given time, which can change over time.
                \end{itemize}
        \end{itemize}
    \end{block}
    The definitions aims to avoid using excessive biological terms but stick to \texttt{observations}, \texttt{state}, \texttt{features}.
\end{frame}

\section{Imaging-based Spatial Transcriptomics and its challenges {[02:30]}}

\begin{frame}
    \frametitle{Imaging-based spatial transcriptomics and its challenges}
    \begin{block}{}
        \begin{itemize}
            \item Thee different technologies to measure spatial transcriptomics (without details).
            \item im-scRNAseq from ISH data can be collected with MERFISH, seqFISH, and other technologies. (without details)
            \item Comparison with scRNAseq and challenges of being limited to hypothesis testing.
        \end{itemize}
    \end{block}
\end{frame}

\begin{frame}
    \frametitle{Integration between im-scRNAseq and scRNAseq}
    \begin{block}{}
        \begin{itemize}
            \item Integration between im-scRNAseq and scRNAseq aims to estimate the location of the signal of cells based
                on the ones that are measured in space.
            \item TANGRAM solves an optimization problem estimating the probability of optimal location 
                of the scRNA-seq cells in space.
            \item TANGRAM is the suggested method for MERFISH data from a benchmark, but requires careful handling.
            \item Reference to a thesis student working on it and argue that some refinement is needed.
        \end{itemize}
    \end{block}
\end{frame}

\section{Exploiting prior knowledge to perform feature selection {[01:30]}}

\begin{frame}
    \frametitle{Biologically informed feature selection}
    \begin{block}{}
        \begin{itemize}
            \item Cells are measured in space, but may differ in their state even if of the same type.
            \item Something must be regulating the state of the cells. Their regulatory mechanism.
            \item THeir regulatory mechanism is stored in Gene Regulatory Networks (GRNs).
            \item We estimate the activity of each regulatory element in space using DecoupleR.
        \end{itemize}
    \end{block}
\end{frame}

\begin{frame}
    \frametitle{Results of spatially emergent patterns from unmeasured features}
    \begin{block}{}
        \begin{itemize}
            \item Results of spatially emergent patterns of colocalized features with biological relevance.
            \item Results of ISMB\/ECCB 2025 poster \/ IJCNN2025 paper.
        \end{itemize}
    \end{block}
\end{frame}

\section{Cell-Cell communication {[01:00]}}

\begin{frame}
    \frametitle{Cell-Cell communication background}
    Now i phrase it in a way that uses biological terms but will eventually be the ones i state at the beginning of the talk.
    \begin{block}{}
        \begin{itemize}
            \item Cell-Cell communication is the process by which cells exchange information.
            \item Analogy with antennas that send and receive at different frequencies.
            \item Analogy with users that subscribe to different social networks, where they can only communicate with those that subscribe to the same network.
        \end{itemize}
    \end{block}
\end{frame}

\section{Complex networks and multilayer networks {[02:00]}}
\begin{frame}
    \frametitle{\scriptsize{Future works: modeling the cell-cell communication with multilayer networks}}
    \begin{block}{}
        \begin{itemize}
            \item Cell-Cell communication is a system of systems, and multilayer networks can model them.
            \item Each layer of the multilayer network represents a different type of communication.
            \item Each node represents a cell, and each edge represents a communication between two cells.
            \item The multilayer network can be used to analyze the structure of the communication system.
        \end{itemize}
    \end{block}
\end{frame}

\begin{frame}
    \frametitle{\scriptsize{Future works: structural analysis and tensor decomposition of the multilayer network}}
    \begin{block}{}
        \begin{itemize}
            \item Structural analysis of the multilayer network can reveal patterns of communication.
            \item Tensor decomposition can be used to extract features from the multilayer network.
            \item Questions to be answered:
                \begin{itemize}
                    \item How do different layers of the multilayer network interact?
                    \item What are the most important nodes in the network?
                    \item How can we use the multilayer network to predict cell-Cell communication?
                \end{itemize}
        \end{itemize}
    \end{block}
\end{frame}

\section{Appendix {[01:00]}}

\begin{frame}
    \frametitle{Appendix: Vignette of N2SIMBA}
    \begin{block}{}
        \begin{itemize}
            \item N2SIMBA is a simulator of bacteria communities driven by a bacteria interaction networks.
            \item bacteria interacts with each other using the cross feeding mechanism.
            \item Interaction networks are unknown and this tool aims to produce realistic synthetic data that reconstruction methods can use to assess their performance.
        \end{itemize}
    \end{block}
\end{frame}

\begin{frame}
    \frametitle{Appendix: Vignette of diversity analysis of piglets}
    \begin{block}{}
        \begin{itemize}
            \item Analysis of the diversity of piglets in a farm that was suspected to present C. Difficile infection.
            \item Results show that the safety guidelines of the farm allowed to avoid the infection.
        \end{itemize}
    \end{block}
\end{frame}

\section{Publications, teaching and courses {[00:15]}}

\begin{frame}
    \frametitle{Publications, teaching and courses}
    \begin{block}{}
        \begin{itemize}
            \item List of publications
            \item Which courses I took
            \item Which courses I assisted
        \end{itemize}
    \end{block}
\end{frame}

\begin{frame}
    \frametitle{Thank you!}
    \begin{block}{}
        \begin{itemize}
            \item Questions?
            \item Contact me at ...
        \end{itemize}
    \end{block}
\end{frame}
